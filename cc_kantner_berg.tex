%    \begin{macrocode}
\documentclass{article}
\usepackage{miniDBV2,amsthm}
\usepackage[a4paper, landscape, margin=5mm]{geometry}
\usepackage{multicol,tabularx}
\renewcommand{\familydefault}{cmss}
\begin{document}

\newtheorem{convention}{}
\newcommand{\Ref}[1]{$^{\ref{#1}}$}

\hspace{2.6mm}
\beginMinikarteNeu
	{\large Henning Kantner}{\large Christoph Berg}
%
\Grundsystem{\large 5er Oberfarben (Rieneck Standard)}
\EinSA{\large15-17}{\large15-17}
%\EinSAkleinesSingle
%\EinSATopSingle
%\EinSAFuenferOFregel
\EinSAFuenferOFselten
\Mindestlaengen{\large3}{\large3}{\large5}{\large5}
\EinTreffBed{3+ \treff}
\EinTreffAnt{Inverted Minors}
	{}
\EinKaroBed{3+ (meist 4+) \karo}
\EinKaroAnt{Inverted Minors}
	{}
\EinCoeurBed{5+ \coeur}
\EinCoeurAnt{2\SA{} Jacoby (PF, guter Fit, danach Werte/2. Farbe, Cuebids)}
	{3\treff/3\karo{} Bergen, 3\coeur/4\coeur{} sperrend}
\EinPikBed{5+ \pik}
\EinPikAnt{dito
	}
		{}
\EinSABed{nat.}
\EinSAAnt{2\treff{} Stayman, Transfers
	(2\pik{} = \treff, 3\treff{} = \karo), alles ab 0F}
	{3\karo/\coeur/\pik Schlemmint., 4\treff{} Assfrage}
%
\ZweiTreffBed
	{Benjamin: bel. Semiforcing (meist mit langer Farbe), starke SA}
	{}
\ZweiTreffAnt{2\karo{} Relais, Rest nat}
\ZweiKaroBed
	{Benjamin: bel. Partieforcing, starke SA}
	{}
\ZweiKaroAnt{2\coeur{} Relais, Rest nat}
\ZweiCoeurBed
	{Weak Two in \coeur\ (ca. 4-10F)}
	{}
\ZweiCoeurAnt{2\SA{} Ogust (min-min-max-max), RONF}
\ZweiPikBed{Weak Two in \pik\ (ca. 4-10F)}
	{}
\ZweiPikAnt{dito}
\ZweiSABed{semiausgeglichen, 20-21F}
	{}
\ZweiSAAnt{3\treff{} Puppet-Stayman, 3\karo/3\coeur{} Transfer
	}
%
\BesondereZweierUndHoeher{3\SA: Gambling (stehende 7er-UF ohne Nebenwerte)}
	{4\SA: 6-5+ in UF}
%
\InfoKontraAb{12}
\InfoKontraOF
%\InfoKontraWerte
\FarbGegenEiner{8}{16}
\FarbGegenZweier{10}{17}
\StilDerGegenreizung{kompetitiv}
\Weiterreizung{Farbwechsel nonforcing}
\EinSAGegen{16-18F}{11-14F}
	{}
\SprungGegen{Weak Jumps}
	{Michaels (�berruf = \pik{} + weitere, 2\SA{} = untere)}
%
\GegenEinSA
	{Multi-Landy: 2\treff: beide OF, 2\karo: OF-Einf�rber, 2\OF: 5-4+ OF-UF, }
	{\hphantom{Multi-Landy:} 2\SA: beide UF, 3\UF: UF-Einf�rber}
	{\hphantom{Multi-Landy:} X ist Strafe}
\AndereGegenreizungen
	{gegen starke \treff: 1\pik "`Habe 13 Karten"', alles h�here Sperrgebote}
	{}
	{}
%
\SequenzHoechste{}
%\SequenzZweite{}
\InnereSequenzHoechste{}
%\InnereSequenzZweite{}
\AusspielDritteFuenfte
%\AusspielVierte
%\AusspielZweiteVierte
\AusspielSonstiges{Double hoch}%, K $\Rightarrow$ L�ngenm.}
\AusspielSA{4.-h�chste, selten 2.}
	{}
%\PositivHoch
\PositivNiedrig
%\PositivSonstiges{}
%\GeradeHoch
\GeradeNiedrig
%\GeradeSonstiges{}
\Abwuerfe{Lavinthal}
\MarkierungenSA{}
	{\hspace{87mm} Vereinbarungen im Innenteil $\rightarrow$}
%
\Datum{\footnotesize \today}
%
\endMinikarteNeu

\begin{twocolumn}

\section*{Vereinbarungen}

\begin{itemize}
\item Supportkontra, Negativkontras, 3. (Unter-)Farbe forcing
\item Lebensohl, selten auch Good-Bad-2NT
\item Assfrage ist generell RKCB 30/41, meist 4\SA; DOPI-ROPI
\item 4 UF ist Assfrage in UF wenn es keine sperrende/kompetitive Situation ist
 (1\treff-4\treff{} ist sperrend, 1\treff-2\treff-3\treff-4\treff{} ist Assfrage)
\item 2/1 ist nach Zwischenreizung nonforcing
\item 1\karo-1\pik-2\karo-3\pik{} ist einladend
\item Nach 2\SA-3\treff-3\karo{} zeigt 4\treff{} beide OF 4-4
\item Nach 1OF und Zwischenreizung ist 2SA nat, �berruf ist mind. einladend
\end{itemize}

\end{twocolumn}

\end{document}
%    \end{macrocode}
