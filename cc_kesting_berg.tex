%    \begin{macrocode}
\documentclass{article}
\usepackage{miniDBV2,amsthm}
\usepackage[a4paper, landscape, margin=5mm]{geometry}
\usepackage{multicol,tabularx}
\renewcommand{\familydefault}{cmss}
\begin{document}

\newtheorem{convention}{}
\newcommand{\Ref}[1]{$^{\ref{#1}}$}

\hspace{2.6mm}
\beginMinikarteNeu
	{\large Hennes Kesting}{\large Christoph Berg}
%
\Grundsystem{\large American Standard}
\EinSA{\large15-17}{\large15-17}
%\EinSAkleinesSingle
%\EinSATopSingle
%\EinSAFuenferOFregel
\EinSAFuenferOFselten
\Mindestlaengen{\large2}{\large4}{\large5}{\large5}
\EinTreffBed{2+ \treff}
\EinTreffAnt{Inverted (auch nach Zwischenreizung)}
	{Generell: Assfrage RKC 30/41, Kontras tendenziell negativ}
\EinKaroBed{4+ \karo}
\EinKaroAnt{dito}
	{}
\EinCoeurBed{5+ \coeur}
\EinCoeurAnt{}
	{}
\EinPikBed{5+ \pik}
\EinPikAnt{ }
	{}
\EinSABed{nat.}
\EinSAAnt{2\treff{} Stayman, OF-Transfers}
	{}
%
\ZweiTreffBed
	{mind. Semiforcing, st�rkste Ansage}
	{}
\ZweiTreffAnt{2\karo{} Relais, Rest nat.}
\ZweiKaroBed
	{Multi (Weak 2 in OF, starke SA)}
	{}
\ZweiKaroAnt{2\coeur{} pass or correct, 2\SA{} fragt (min \coeur/max \coeur/min \pik/max \pik)}
\ZweiCoeurBed
	{Zweif�rber mit \coeur}
	{}
\ZweiCoeurAnt{}
\ZweiPikBed{Zweif�rber \pik{} und UF}
	{}
\ZweiPikAnt{}
\ZweiSABed{UF-Zweif�rber}
	{}
\ZweiSAAnt{
	}
%
\BesondereZweierUndHoeher{3\SA: Gambling (stehende 7er-UF ohne Nebenwerte)}
	{4\SA: 6-5+ in UF}
%
\InfoKontraAb{12}
\InfoKontraOF
%\InfoKontraWerte
\FarbGegenEiner{8}{16}
\FarbGegenZweier{10}{17}
\StilDerGegenreizung{kompetitiv bis konstruktiv}
\Weiterreizung{Farbwechsel nonforcing}
\EinSAGegen{16-18F}{11-14F}
	{semiausgeglichen}
\SprungGegen{Weak Jumps}
	{Schr�der (�berruf = andere UF bzw. OF, 2\SA{} = untere)}
%
\GegenEinSA
	{DONT (X = Einf�rber)}
	{}
	{}
\AndereGegenreizungen
	%{gegen 2\karo-Multi: X: Info-Kontra gegen \coeur-Weak Two}
	%{Lebensohl nach Info-X; 2\SA\ als Zweif�rber in 4. Hand und Good-Bad NT}
	%{gegen starke \treff: 1x nat, 2x/3x wie gegen 1\SA}
	{}
	{}
	{}
%
\SequenzHoechste{}
%\SequenzZweite{}
%\InnereSequenzHoechste{}
\InnereSequenzZweite{}
\AusspielDritteFuenfte
%\AusspielVierte
%\AusspielZweiteVierte
\AusspielSonstiges{Klein zeigt Figur}
\AusspielSA{4.-h�chste}
	{}
%\PositivHoch
\PositivNiedrig
%\PositivSonstiges{}
%\GeradeHoch
\GeradeNiedrig
%\GeradeSonstiges{}
\Abwuerfe{Lavinthal}
\MarkierungenSA{}
	{}
	%{\hspace{87mm} Vereinbarungen im Innenteil $\rightarrow$}
%
\Datum{\footnotesize \today}
%
\endMinikarteNeu

%\begin{twocolumn}
%
%\section*{Vereinbarungen}
%
%\begin{itemize}
%\item Supportkontra, Negativkontras, 3. (Unter-)Farbe forcing
%\item Assfrage ist generell RKCB 30/41
%\item 2/1 ist nach Zwischenreizung nonforcing
%\end{itemize}
%
%\end{twocolumn}

\end{document}
%    \end{macrocode}
