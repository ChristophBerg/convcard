%    \begin{macrocode}
\documentclass{article}
\usepackage{miniDBV2}
\usepackage[a4paper, landscape, margin=5mm]{geometry}
\usepackage{multicol,tabularx,amssymb}
\renewcommand{\familydefault}{cmss}

\newcounter{conv}
\setcounter{conv}{0}
\newenvironment{convention}[2]{{\bf \refstepcounter{conv}\label{#2}\theconv\ #1.}}{}
\newcommand{\Ref}[1]{$^{\ref{#1}}$}
\newcommand{\s}{$\leadsto$}
\renewcommand{\t}{$\cdot$}
\newcommand{\se}{;\hspace{1.5ex}}

\begin{document}

\hspace{2.6mm}
\beginMinikarteNeu
	{}{}
%
\Grundsystem{{\large Moscito 2005}}
\EinSA{\large 11-14 FP}{\large 11-14 FP}
%\EinSAkleinesSingle
%\EinSATopSingle
%\EinSAFuenferOFregel
%\EinSAFuenferOFselten
\Mindestlaengen{---}{$\!$4\coeur}{$\!$4\pik}{$\!$UF}
\EinTreffBed{15+FP/9+QP, bel. Verteilung}
\EinTreffAnt{1\karo, 2\SA{}+: stark, 9+FP/6+QP, 1\pik: schwach, 0-5FP/0-2QP, bel.}
	{Rest: semipositiv, 6-8FP, 3-5QP. Details umseitig}
\EinKaroBed{4+\coeur, 9-14FP/6-8QP, 5er-UF m�glich, wenn ausgegl. Tendenz 13-14FP}
\EinKaroAnt{1\coeur: Step, 2\coeur, 3\coeur: sperrend, 2\SA: 4\coeur, einl., 3\SA: \coeur-Fit, Spielvorschlag}
	{Rest: nat., 2/1 ist 1-forcing}
\EinCoeurBed{4+\pik, analog 1\karo}
\EinCoeurAnt{Analog 1\karo-Antworten}
	{}
\EinPikBed{6+\karo (evtl. 4er-OF) oder UF-Zweif�rber, 9-14FP/6-8QP}
\EinPikAnt{1\SA: Step, 2\treff/\karo: 5+ \coeur/\pik, 2\coeur: 5+\treff, 2\pik: 3-3+UF, 2\SA: \karo-Hebung,}
	{3\treff: pass/correct, 3\karo: zum Spielen}
\EinSABed{ausgeglichen, 11-14 FP, wenn 4er-OF, dann 11-12}
\EinSAAnt{2\treff: Stayman (0\pl FP), 2\karo/\coeur: Transfer \ra OF (0\pl FP),
	2\pik: Transfer \ra UF (0\pl FP)}
	{4\treff: Gerber (04/1/2/3), 4\karo: 5-5\pl \OF}
%
\ZweiTreffBed{6+\treff (keine \karo, evtl. 4er-OF), 9-14FP/6-8QP}
	{}
\ZweiTreffAnt{2\karo: Step, 2\SA: einl., Rest nonforcing}
\ZweiKaroBed{6er-OF, 5-10FP}
	{}
\ZweiKaroAnt{}
\ZweiCoeurBed{5\coeur{}(332), 10-12FP}
	{}
\ZweiCoeurAnt{2\SA: Frage nach Double, Rest nonforcing}
\ZweiPikBed{5\pik{}(332), analog 2\coeur}
	{}
\ZweiPikAnt{Analog 2\coeur-Antworten}
\ZweiSABed{}
	{}
\ZweiSAAnt{}
%
\BesondereZweierUndHoeher{3\UF: 7\pl \UF (5-10 FP, in 3. Hand 6er m�glich),
	3\SA: Gambling (in 3./4. Hand zum Spielen)}
	{4\UF: stehendes 7er-\OF\ mit Nebenwert, 4\SA: 6-5\pl in \UF}
%
%
% Gegenreizung und Verteidigung
% 5er-OF (cc_walsh.tex) und Moscito (cc_moscito05.tex)
%
\InfoKontraAb{12 FP}
\InfoKontraOF
%\InfoKontraWerte
\FarbGegenEiner{8}{15}
\FarbGegenZweier{10}{16}
\StilDerGegenreizung{in der Regel konstruktiv, schwache Hebungen}
\Weiterreizung{Farbw.\ nonf., �berr.\ fr.\ Stopper ($\rightarrow$\SA),
	zeigt Fit/stark Einf.}
\EinSAGegen
	%{15-18 FP}
	{Raptor}
	{11-14 FP}
	{relativ ausgeglichen, Stopper, Weiterreizung wie nach 1 \SA-Er�ffnung}
\SprungGegen{Weak Jumps: schwach, sperrend}
	{Michaels Pr�zis:
	(1UF)\ra 2\karo: \OF, 2\SA: aUF/\coeur;
	(1OF)\ra 2OF: \treff/aOF, 2\SA: \UF, 3\treff: \karo/aOF}
%
\GegenEinSA{Multi-Landy: X: 5-4 UF-OF, 2\treff: 5-4\pl \OF\ (\ra 2\karo: gleiche L�nge in \OF),
	2\karo: OF-Einf�rber,}
	{\hphantom{Multi-Landy:} 2OF: 5-4\pl OF/UF, 2\SA: 5-5\pl in \UF, 3UF: UF-Einf�rber}
	{gegen schwachen \SA: X: mind. gleiche St�rke (oberes Ende Punktspanne), Rest wie oben}
\AndereGegenreizungen
	%{Starke 1\treff \ra Crash}
	{Starke 1\treff \ra Multi-Landy (1\SA: UF, X: 13-15 ausgegl.)}
	{2\karo-Multi \ra X: Info-X gegen \coeur-Weak Two}
	{Lebensohl und Scrambling 2\SA\ nach Info-X;
	1\SA/2\SA\ als Zweif�rber in 4. Hand; Kompetitive 2\SA}
%
\SequenzHoechste{}
%\SequenzZweite{}
%\InnereSequenzHoechste{}
\InnereSequenzZweite{}
\AusspielDritteFuenfte
%\AusspielVierte
%\AusspielZweiteVierte
\AusspielSonstiges{Double hoch}
\AusspielSA{4.-h�chste, 10/9 verspricht 0 oder 2 h�here}
	{}
%\PositivHoch
\PositivNiedrig
%\PositivSonstiges{}
%\GeradeHoch
\GeradeNiedrig
%\GeradeSonstiges{}
\Abwuerfe{Direkte Marke, Figur zeigt Karte darunter}
\MarkierungenSA{Lavinthal, Smith Peter}
	{}
%

%
\Datum{\today}
%
\endMinikarteNeu

\newpage
\footnotesize

\setlength{\parindent}{0pt}
\setlength{\parskip}{1ex}
%\setlength{\columnseprule}{.4pt}

\begin{multicols*}{4}

\section*{Systembeschreibung}

Das n�chste Gebot ("`Step"', \s) fragt jeweils nach weiterer Blattbeschreibung,
sonst wird nat�rlich (\t) weitergereizt. Es werden Blatttyp, dann K�rzen, genaue
Verteilung, St�rke in QP �ber bekannter Basis, und Denial Cuebids erfragt
(�berspringen: Keycard Blackwood). 3\SA ist zum Spielen, 4\karo das
Endsignal der Relais.

\begin{convention}{Punktspannen}{points} Er�ffnung: Limit: 9-14FP, 6-8QP, 1\Cl:
15+FP, 9+QP, 1\Cl-1\Di-1\He: 18+FP, 12+QP. Antworten auf 1\Cl: 1\Sp schwach:
0-5FP/0-2QP, semi-positiv: 6-9FP, 3-5QP, 1\Di/2\SA{}+ stark: 9+FP, 6+QP.

\end{convention}

\begin{convention}{Legende}{legende} ~

\begin{tabularx}{\columnwidth}{|lX|}
\hline
(54)04 & 5-4-0-4 oder 4-5-0-4 \\
4441* & 1444, 4144, 4414, 4441 (numerisch) \\
\pik, \pik{}+\coeur & \pik-Einf�rber, \pik{}/\coeur-Zweif�rber \\
R, NR, LL & revers, nicht-revers, long-legged-Zweif�rber \\
\hline
\end{tabularx}
\end{convention}

\begin{convention}{1\treff--1\karo}{1tr2ka} Positiv.
Mit 12+QP (18+FP) fragt der Er�ffner mit 1\coeur
weiter, mit Minimum zeigt er selbst.

\begin{tabularx}{\columnwidth}{|lX|}
\hline
1\treff--1\karo & weiter \t direkt oder \s Step \\
1\pik   & Zweif. ohne \pik{} / UF-Einf. \\
\quad \s 2\treff & \treff{}+\coeur{} / \treff \\
\quad \quad \s 2\coeur & \treff \\
\quad \quad \s 2\pik{}+ & \treff{}+\coeur \\
\quad \s 2\karo & \karo{}+\coeur \\
\quad \s 2\coeur & \karo \\
\quad \s 2\pik{}+ & UF-Zweif�rber \\
1\SA    & ausgegl. / 4441 \\
\quad \s 2\karo & keine OF \\
\quad \s 2\coeur & ausgegl., 4-5 \coeur \\
\quad \s 2\pik & ausgegl., 4 \pik \\
\quad \s 2\SA & 5\pik{}(332) \\
\quad \s 3\treff & 44(32) \\
\quad \s 3\karo..3\SA & 4441* \\
\quad \t 2\karo & Frage nach OF (Puppet-artig) \\
\quad \quad \t 2\coeur..3\treff & 4441* \\
\quad \quad \t 3\karo & 4OF(432,333) \\
\quad \quad \t 3\coeur,3\pik & 5OF(332) \\
\quad \quad \t 3\SA & keine 4er-OF \\
2\treff & \treff{}+\pik{} / \pik \\
\quad \s 2\coeur & \pik \\
\quad \s 2\pik{}+ & \treff{}+\pik \\
2\karo  & \karo{}+\pik \\
2\coeur & \coeur \\
2\pik{}+ & OF-Zweif�rber \\
\hline
\end{tabularx}
\end{convention}

\begin{convention}{1\treff--2\SA{}+}{1tr3f} Positiv, Dreif�rber mit Chicane oder
stehende Farbe.

\begin{tabularx}{\columnwidth}{|lX|}
\hline
1\treff & \\
\t 2\SA & bel. 5440 mit OF-Chicane \\
\t 3\treff & stehende 7er oder 8er Farbe \\
\t 3\karo & bel. 5440 mit \karo-Chicane \\
\t 3\coeur..3\SA & (445)0 \\
\hline
\end{tabularx}
\end{convention}

\vfill
\columnbreak

\begin{convention}{1\treff--1\coeur,1\SA,2\treff,\dots,2\pik}{sempos}
Semipositiv.

\begin{tabularx}{\columnwidth}{|lX|}
\hline
1\treff--1\coeur   & ausgegl. / unausgegl. ohne 5er OF \\
\s 1\SA & ausgegl. \\
\s 2\treff & \Cl{} / \coeur{}+\treff{} / \pik{}+\treff \\
\quad \s 2\coeur & \coeur{}+\treff (R) \\
\quad \s 2\pik & \treff \\
\quad \s 2\SA{}+ & \pik{}+\treff (R) \\
\s 2\karo & UF-Zweif�rber \\
\s 2\coeur & \coeur{}+\karo (R) \\
\s 2\pik & \pik{}+\karo (R) \\
\s 2\SA & 04(45) \\
\s 3\treff & 40(45) \\
\s 3\karo & 4405 \\
\s 3\coeur & 4450 \\
\t 1\SA & 15-17FP ausgegl. \\
\t 2\treff & 5+\coeur \\
\t 2\karo & 5+\pik \\
\t 2\coeur & \coeur{}+UF (R) \\
\t 2\pik & \pik{}+UF (R) \\
\t 2\SA & UF-Zweif�rber \\
\hline
1\treff--1\SA    & \karo{} / 5\coeur{}+UF / 5OF(440) \\
\s 2\karo & \coeur{}+\treff (NR) \\
\s 2\coeur & \coeur{}+\karo (NR) \\
\s 2\pik & \karo \\
\s 2\SA & (54)04 \\
\s 3\treff & (54)40 \\
\s 3\karo & 0544 \\
\s 3\coeur & 5044 \\
\t 2\karo,2\coeur & pass/correct \\
\t 2\pik & nat., nf \\
\t 2\SA & \coeur \\
\t 3\treff,3\karo & pass/correct \\
\hline
1\treff--2\treff & \coeur{} / 5\pik{}+UF \\
\s 2\coeur & \pik{}+\treff (NR) \\
\s 2\pik & \coeur \\
\s 2\SA{}+ & \pik{}+\karo (NR) \\
\t 2\coeur,2\pik & pass/correct \\
\t 2\SA & \pik \\
\t 3\treff..3\coeur & nat., nf \\
\hline
1\treff--2\karo  & 5\pik-4\coeur(NR oder LL) \\
\t 2\coeur & Step \ra Zweif�rber \\
\t 2\pik & zum Spielen \\
\t 2\SA..3\coeur & nat., nf \\
\hline
1\treff--2\coeur & 4\pik-5+\coeur(R) \\
\t 2\pik & Pr�ferenz (!) \\
\t 2\SA & Step (!) \ra Zweif�rber \\
\t 3\treff,3\karo & nat., nf \\
\hline
1\treff--2\pik & \pik \\
\t 2\SA & Step \ra Einf�rber \\
\t 3\treff..3\coeur & nat., nf \\
\hline
\end{tabularx}
\end{convention}

\begin{convention}{2\coeur,2\pik}{2of} 10-12FP, 5OF(332).

\begin{tabularx}{\columnwidth}{|lX|}
\hline
2\coeur,2\pik & \\
\t 2\SA & Frage nach Double \\
\t Rest & 100\% nonforcing \\
\hline
\end{tabularx}
\end{convention}

\vfill
\columnbreak

\begin{convention}{Einf�rber}{1suiter} Aufl�sung der Einf�rber.

\begin{tabularx}{\columnwidth}{|rcccX|}
\hline
  & stark & semi & & \\
  & 2\He{}\s & 2\Sp{}\s & & \\
1 & 2\SA & 3\Cl & keine & $\downarrow$ \\
2 & 3\Cl & 3\Di & hohe & 6223 \\
3 & 3\Di & 3\He & mittl. & 6232 \\
4 & 3\He & 3\Sp & 6331 & 6322 \\
5 & 3\Sp & 3\SA & 7321 & 7222 \\
6 & 3\SA & 4\Cl & 7330 & \\
\hline
\end{tabularx}
\end{convention}

\begin{convention}{Zweif�rber}{2suiter} Aufl�sung der Zweif�rber.

\begin{tabularx}{\columnwidth}{|rcccX|}
\hline
  & stark & semi & & \\
  & 2\Di{}\s & 2\He{}\s & & \\
1 & 2\Sp & 2\SA & R & \\
2 & 2\SA & 3\Cl & LL & $\downarrow$ \\
3 & 3\Cl & 3\Di & hohe & \\
4 & 3\Di & 3\He & 5422 & 5521 \\
5 & 3\He & 3\Sp & 5431 & 5530 \\
6 & 3\Sp & 3\SA & 6421 & 6511 \\
7 & 3\SA & 4\Cl & 6430 & 6520 \\
8 & 4\Cl & 4\Di & 7411 & 6610 \\
9 & 4\Di & 4\He & 7420 & 7510 \\
10&  &      &      & 7600 \\
\hline
\end{tabularx}
\end{convention}

\begin{convention}{1\karo}{1ka} 9-14FP/6-8QP, 4+\coeur.

\begin{tabularx}{\columnwidth}{|lX|}
\hline
1\karo & \\
\s 1\pik & \pik{}-\coeur \\
\quad \s 2\treff & \pik{}-\coeur{}-\treff \\
\quad \quad \s 2\coeur..2\SA & 4414, 4405, 4504 \\
\quad \s 2\karo & \pik{}-\coeur{}-\karo \\
\quad \quad \s 2\pik..3\treff & 4441, 4450, 4540 \\
\quad \s 2\coeur & 44(32) \\
\quad \s 2\pik & 5\pik-6\coeur (LL, R) \\
\quad \s 2\SA{}+ & 4\pik-5\coeur (R) \\
\s 1\SA & ausgegl. / Dreif. o. \pik{} / 4\coeur-5\karo \\
\quad \s 2\karo & 4\coeur-5\karo (R, Fehler?) \\
\quad \s 2\coeur & ausgegl., 4\coeur \\
\quad \quad \s 2\SA & 4\coeur-4\treff-32 \\
\quad \quad \s 3\treff,3\karo & 2443, 3442 \\
\quad \quad \s 3\coeur & 3433(!) \\
\quad \s 2\pik & 5\coeur{}(332) \\
\quad \s 2\SA & Dreif�rber ohne \pik \\
\quad \quad \s 3\treff..3\pik & 1444, 0445, 0454, 0544 \\
\s 2\treff & \treff (4-5 m�gl.) \\
\s 2\karo & \coeur{}+\karo (NR) \\
\s 2\coeur & \coeur-Einf�rber, min. \\
\s 2\pik{}+ & \coeur-Einf�rber, max. \\
\t 1\pik & 4+\pik, 6-12FP \\
\t 1\SA & ausg., 6-9FP \\
\t 2\SA & 4\coeur, einl. (* neue Farbe: K�rze) \\
\t 3\coeur & 4\coeur, 6-9FP \\
\t 3\SA & \coeur-Fit, 12-15FP, ausg. (*) \\
\hline
\end{tabularx}
\end{convention}

\vfill
\columnbreak

\begin{convention}{1\coeur}{1co} 9-14FP/6-8QP, 4+\pik.

\begin{tabularx}{\columnwidth}{|lX|}
\hline
1\coeur & \\
\s 1\SA & ausgegl. / Dreif. o. \coeur{} / 4\pik-5\karo \\
\quad \s 2\karo & 4\pik-5\karo (R, Fehler?) \\
\quad \s 2\coeur & ausgegl., 4\pik, keine 4\coeur \\
\quad \quad \s 2\SA & 4\pik-4\treff-32 \\
\quad \quad \s 3\treff,3\karo & 4243, 4342 \\
\quad \quad \s 3\coeur & 4333(!) \\
\quad \s 2\pik & 5\pik{}(332) \\
\quad \s 2\SA & Dreif�rber ohne \coeur \\
\quad \quad \s 3\karo..3\SA & 4144, 4045, 4054, 5044 \\
\quad \s 3\treff,3\karo & 5404, 5440 \\
\s 2\treff & \treff (4-5 m�gl.) \\
\s 2\karo & \pik{}+\karo (NR) \\
\s 2\coeur & \pik{}+\coeur (NR) \\
\s 2\pik & \pik-Einf�rber, min. \\
\s 2\SA{}+ & \pik-Einf�rber, max. \\
\t 1\SA & ausg., 6-9FP \\
\t 2\SA & 4\pik, einl. (*) \\
\t 3\pik & 4\pik, 6-9FP \\
\t 3\SA & \pik-Fit, 12-15FP, ausg. (*) \\
\hline
\end{tabularx}
\end{convention}

\begin{convention}{1\pik}{1pi} 9-14FP/6-8QP, \karo-Einf�rber oder UF-Zweif�rber.

\begin{tabularx}{\columnwidth}{|lX|}
\hline
1\pik & \\
\s 2\treff & UF-Zweif�rber \\
\s 2\karo & \karo-Einf�rber, min. \\
\s 2\coeur & 6\karo-4\coeur \\
\s 2\pik & 6\karo-4\pik \\
\s 2\SA{}+ & \karo-Einf�rber, max. \\
\t 2\treff & 5+\coeur Transfer: Annahme: Fx oder \dots \\
\t 2\karo & 5+\pik \dots besser, Step = negativ \\
\quad \t 2SA & Partieinteresse \\
\quad \quad \t 3\treff,3\karo & lange Farbe, min. \\
\quad \quad \t 3OF & 3er OF, min. \\
\quad \quad \t 3aOF & lange \treff, max., keine 3er OF \\
\quad \quad \t 3\SA & lange \karo, max. keine 3er OF \\
\quad \quad \t 4OF & 3er OF, max. \\
\t 2\coeur & 5+\treff \\
\t 2\pik & Limithebung in jeder UF (3+-3+)\\
\quad \t 2\SA & \treff, max. \\
\quad \t 3\treff,3\karo & zum Spielen \\
\quad \t 3\coeur,3\pik & \karo, max., K�rze \\
\t 2\SA & 4+\karo, Limithebung \\
\t 3\treff & pass/correct \\
\t 3\karo & zum Spielen \\
\hline
\end{tabularx}
\end{convention}

\begin{convention}{2\treff}{2tr} 9-14FP/6-8QP, \treff-Einf�rber, evtl. 4er-OF.

\begin{tabularx}{\columnwidth}{|lX|}
\hline
2\treff & \\
\s 2\coeur & 6\treff-4\coeur \\
\s 2\pik & 6\treff-4\pik \\
\s 2\SA{}+ & \treff-Einf�rber \\
\t 2\SA & Einladung zum Vollspiel \\
\quad \t 3\treff & min. \\
\quad \t 3\coeur..4\treff & max., K�rze \\
\quad \t 3\SA & max. \\
\t Rest & Nonforcing \\
\hline
\end{tabularx}
\end{convention}

\end{multicols*}

\end{document}
%    \end{macrocode}
