\documentclass{article}
\usepackage[utf8x]{inputenc}
\usepackage[T1]{fontenc}
\DeclareUnicodeCharacter{9829}{$\heartsuit$}
\DeclareUnicodeCharacter{9830}{$\diamondsuit$}
\usepackage{amsthm}
\usepackage{german}
\usepackage[a4paper, landscape, left=0mm,top=5mm,bottom=5mm, right=10mm]{geometry}
\usepackage{multicol,tabularx,rotating}
\renewcommand{\familydefault}{cmss}
\begin{document}
\begin{multicols}{3}
\raggedcolumns

\begin{tabularx}{\columnwidth}{|X|}
\hline \multicolumn{1}{c}{\bf \large Gegenreizung und kompetitive Reizung} \\
\hline \bf Überrufe (Stil, Antworten, Reopening) \\
 aggressiv bis konstruktiv \\
\\
\\
\\
\\
\\
\hline \bf 1SA Überruf (2./4. Position, Antworten, Reopen) \\
 16-18F \\
 weiter mit System on \\
 (1x)-p-(1y)-1SA ist nat \\
\\
\\
\hline \bf Sprunggegenreizung (Stil, Antworten, Unusual NT) \\
 Michaels: \\
 1UF-2♦ = OF, 1OF-2OF = andere OF, 2SA = untere \\
\\
\\
\hline \bf Cue-Bid, Sprung Cue-Bid (Stil, Antworten, Reopen) \\
\\
\\
\\
\\
\hline \bf Gegen 1 SA (stark, schwach, 2./4. Hand) \\
 Multi-Landy \\
 2♣: beide OF, 2♦: 6er OF, 2OF: 5OF-4UF, 3UF: 6er \\
\\
\hline \bf Gegen Sperransagen (Kontras, Cue-Bids, Sprünge) \\
\\
\\
\\
\\
\\
\\
\\
\hline \bf Gegen starke Treff und andere künstl. Eröffnungen \\
\\
\\
\\
\\
\hline \bf Nach Negativ-Kontra des Gegners \\
 Nach 1OF-(X) ist 2UF System on \\
 1SA-2♦-(X)-2♥ zeigt 3er ♥ \\
\\
\\
\hline \end{tabularx}

\begin{tabularx}{\columnwidth}{|l|X|X|}
\hline \multicolumn{3}{c}{\bf \large Ausspiele und Markierung} \\

\hline \multicolumn{3}{c}{\bf Ausspiele (grundsätzlich)} \\
\hline           & {\bf Ausspiel} & {\bf In Partners Farbe} \\
\hline Farbe     & 2./4. & 2./4. \\
\hline SA        & 2./4. & 2./4. \\
\hline Nachfolg. & Länge & Länge \\
\hline \multicolumn{3}{|l|}{\bf Andere:} \\
       \multicolumn{3}{|l|}{\underline{F}x, B10x\underline{x}, B\underline{9}xx} \\

\hline \multicolumn{3}{c}{\bf Ausspiele} \\
\hline {\bf Ausspiel} & {\bf Gegen Farbkontr.} & {\bf Gegen SA} \\
\hline As       & & \\
\hline König    & & \\
\hline Dame     & & \\
\hline Bube     & & \\
\hline 10       & \underline{10}x & klein von 10x \\
\hline 9        & & \\
\hline Hoch-x   & ungerade Länge & ungerade Länge \\
\hline Klein-x  & gerade Länge & gerade Länge \\
\hline \end{tabularx}

\begin{tabularx}{\columnwidth}{|l|X|X|X|}
 \multicolumn{4}{c}{\bf Reihenfolge der Markierung} \\
\hline                & {\bf Partner} & {\bf Gegner} & {\bf Abwurf} \\
\hline \bf Farbe \hfill 1. & Attitude  & Länge     & Lavinthal \\
\hline \bf       \hfill 2. & Länge     & Lavinthal & \\
\hline \bf       \hfill 3. & Lavinthal &           & \\
\hline \bf SA    \hfill 1. & Attitude  & Länge     & Lavinthal \\
\hline \bf       \hfill 2. & Länge     & Lavinthal & \\
\hline \bf       \hfill 3. & Lavinthal &           & \\
\hline \multicolumn{4}{|l|}{\bf Markierungen (inklusive Trumpffarbe):} \\
       \multicolumn{4}{|l|}{niedrig-hoch} \\
       \multicolumn{4}{|l|}{} \\
\hline
\end{tabularx}

\begin{tabularx}{\columnwidth}{|X|}
\multicolumn{1}{c}{\bf \large Kontras} \\
\hline {\bf Informationskontra (Stil; Antworten; Reopening)} \\
       {aggressive Reopenings, auch pre-balancing} \\
       {} \\
       {} \\
       {} \\
\hline {\bf Negativ-Kontra, Kompetitiv-Kontra und weitere} \\
       {\bf (Re-)Kontras} \\
       {Negativ-X bis 4♥} \\
       {zeigt (andere) OF, UF kann auch kurz sein} \\
       {Bsp: 1♦-(1♠)-X zeigt ♥, nicht unbedingt ♣} \\
       {} \\
       {} \\
       {} \\
       {} \\
       {} \\
\hline \end{tabularx}

\begin{tabularx}{\columnwidth}{|X|}
\hline \multicolumn{1}{c}{\bf \Large Deutsche Konventionskarte} \\
\hline \multicolumn{1}{c}{\bf \Large ♠ ♥ DBV e.V. ♦ ♣} \\
\hline {\bf Kategorie:} B \\
\hline {\bf Club:} BC Bayer Leverkusen - II \\
\hline {\bf Turnier:} Rhein-Ruhr Liga 2012 \\
\hline {\bf Paar:} Michael Goetze -- Christoph Berg \\

\hline \multicolumn{1}{c}{\bf \Large System-Zusammenfassung} \\
\hline {\bf Genereller Stil} \\
 5er Oberfarben \\
 Game tries sind nat \\
 Bergen raises \\
 \\
\hline {\bf 1 SA Eröffnung} \\
 15-17 bal \\
\hline {\bf 2 über 1 Antworten} \\
 selbstforcierend \\
\hline {\bf Gebote, die besondere Gegenreizungen erfordern} \\
 2♦ = Wilkosz (Zweifärber 5/5+ mit Oberfarbe) \\
 \\
 \\
 \\
 \\
 \\
 \\
 \\
 \\
 \\
 \\
 \\
 \\
 \\
 \\
 \\
 \\
 \\
\hline {\bf Forcing Pass Sequenzen} \\
 \\
 \\
 \\
\hline {\bf Wichtige sonstige Bemerkungen} \\
 \\
 \\
 \\
\hline {\bf Bluffs} \\
 keine bekannt \\
\hline \end{tabularx}
\begin{flushright}
{\scriptsize \today}
\end{flushright}

\end{multicols}

\begin{tabularx}{\columnwidth}{|c|c|c|c|l|l|X|l|}
\hline
 \begin{sideways}\bf Eröffnung\end{sideways} &
 \begin{sideways}X\,wenn\,künstlich\end{sideways} &
 \begin{sideways}Min.\,Anz.\,Karten\end{sideways} &
 \begin{sideways}Negativ-X\,bis\end{sideways} &
 \bf Beschreibung &
 \bf Antworten &
 \bf Weiterreizung &
 \bf Änderungen als gepasste Hand \\
\hline 1♣   & &3   & & & Weak Jumps & & \\
\hline 1♦   & &3   & & & dto & & \\
\hline 1♥   & &5   & & & Bergen, Splinter, dto & & \\
\hline 1♠   & &5   & & & dto & 1♠-1SA-3♥ ist F & \\
\hline 1 SA & &    & & & & & \\
\hline 2♣   & &    & & PF, SF, SA 22+ & 2♦ neg & & \\
\hline 2♦   &X&0   & & Zweifärber mit einer OF & & & \\
\hline 2♥   & &6   & & ♥-Weak Two & & & \\
\hline 2♠   & &6   & & ♠-Weak Two & & & \\
\hline 2 SA & &    & & 20-21 & Puppet-Stayman & & \\
\hline 3♣   & &7   & & & & & \\
\hline 3♦   & &7   & & & & & \\
\hline 3♥   & &7   & & & & & \\
\hline 3♠   & &7   & & & & & \\
\hline 3 SA &X&7   & & Gambling & & \multicolumn{2}{l|}{\bf Gebote auf hoher Stufe (inkl. Schlemmreizung)} \\
\hline 4♣   & &8   & & & & \multicolumn{2}{l|}{RKCB 1430} \\
\hline 4♦   & &8   & & & & \multicolumn{2}{l|}{} \\
\hline 4♥   & &8(7)& & & & \multicolumn{2}{l|}{} \\
\hline 4♠   & &8(7)& & & & \multicolumn{2}{l|}{} \\
\hline \end{tabularx}

\end{document}
