\documentclass{article}
\usepackage[utf8x]{inputenc}
\usepackage[T1]{fontenc}
\DeclareUnicodeCharacter{9829}{$\heartsuit$}
\DeclareUnicodeCharacter{9830}{$\diamondsuit$}
\usepackage{amsthm}
\usepackage{german}
\usepackage[a4paper, landscape, left=0mm,top=5mm,bottom=5mm, right=10mm]{geometry}
\usepackage{multicol,tabularx,rotating}
%\renewcommand{\familydefault}{cmss}
\usepackage{txfonts}
\setlength{\columnsep}{8mm}

\begin{document}
\begin{multicols}{3}
\raggedcolumns

\begin{tabularx}{\columnwidth}{|X|}
\hline \multicolumn{1}{|c|}{\bf \large Gegenreizung und kompetitive Reizung} \\
\hline \bf Überrufe (Stil, Antworten, Reopening) \\
\hline aggressiv bis konstruktiv \\
 Sprünge in Weiterreizung sind Fit-Jumps \\
 Cuebid zeigt Fit, mind. einladend\\
 Neue Farbe forcing\\
\\
\\
\hline \bf 1SA Überruf (2./4. Position, Antworten, Reopen) \\
\hline 15-18, danach System on \\
 (1x)-p-(1y)-1SA ist nat \\
 Reopening 11-16\\
\\
\\
\hline \bf Sprunggegenreizung (Stil, Antworten, Unusual NT) \\
\hline 1♣-2♦ beide OF, 1x-2SA die beiden untersten Farben \\
Ansonsten schwache Sprünge.
\\
\\
\hline \bf Cue-Bid, Sprung Cue-Bid (Stil, Antworten, Reopen) \\
\hline 1♦-2♦ beide OF; 1OF-2OF andere OF mit einer UF (2SA fragt)\\
1♣-2♣ ist nat\\
\\
\\
\hline \bf Gegen 1 SA (stark, schwach, 2./4. Hand) \\
\hline Kontra zeigt Punkte wenn 13F oder weniger enthalten, sonst 5er UF mit 4er OF (2♣ p/c, 2♦ fragt OF), 2♣ beide OF ab 5-4 (2♦ fragt längere), 2♦ 6er OF, 2OF: 5er mit 4er UF, 2SA beide UF \\
\\
\hline \bf Gegen Sperransagen (Kontras, Cue-Bids, Sprünge) \\
\hline Lebensohl nach t/o von W2 (auch reopening, auch angepasst)\\
2OF-4UF zeigt UF und andere OF\\
\\
\\
\hline \bf Gegen starke Treff und andere künstl. Eröffnungen \\
\hline Gegen 1♣ stark: Kontra rote/schwarze, 1♦ OF/UF, 1SA spitze/runde\\
Gegen 2♦ Multi: Kontra t/o gegen Pik, 2♥ t/o gegen C\oe{}ur\\
\\
\hline \bf Nach Negativ-Kontra des Gegners \\
\hline Nach 1OF-(X) System on (inkl. Bergen) \\
 1SA-2♦-(X)-2♥ zeigt 3er ♥ \\
\\
\\
\hline \end{tabularx}

\begin{tabularx}{\columnwidth}{|l|X|X|}
\hline \multicolumn{3}{|c|}{\bf \large Ausspiele und Markierung} \\

\hline \multicolumn{3}{|c|}{\bf Ausspiele (grundsätzlich)} \\
\hline           & {\bf Ausspiel} & {\bf In Partners Farbe} \\
\hline Farbe     & 2./4. & 2./4. \\
\hline SA        & 2./4. & 2./4. \\
\hline Nachfolg. & Länge & Länge \\
\hline Andere    & \multicolumn{2}{|l|}{} \\

\hline \multicolumn{3}{|c|}{\bf Ausspiele} \\
\hline {\bf Ausspiel} & {\bf Gegen Farbkontr.} & {\bf Gegen SA} \\
\hline As       & \multicolumn{2}{l|}{Ax(+), AH(+), AKQ(+)} \\
\hline König    & \multicolumn{2}{l|}{Kx(+), KH(+), KQJ(+) } \\
\hline Dame     & \multicolumn{2}{l|}{(A)Qx, (A)QJx(+), (A)QJT(+)} \\
\hline Bube     & \multicolumn{2}{l|}{Jx, (K)JT(+), (A)JT(+)} \\
\hline 10       & HT9(+), HTx & HT9(+), HTx, Tx \\
\hline 9        & J9x(+), T9x(+) & J9x(+), T9x(+) \\
\hline Hoch-x   & xS(+), HSx & xS(+), HSx, TSx \\
\hline Klein-x  & HxxS(+), xS & HxxS(+), xS, TS \\
\hline \end{tabularx}

\begin{tabularx}{\columnwidth}{|l|X|X|X|}
 \multicolumn{4}{|c|}{\bf Reihenfolge der Markierung} \\
\hline                & {\bf Partner} & {\bf Gegner} & {\bf Abwurf} \\
\hline \bf Farbe \hfill 1. & Lo = Pos  & Lo = Even & SP \\
\hline \bf       \hfill 2. & Lo = Even & SP        & \\
\hline \bf       \hfill 3. & SP        &           & \\
\hline \bf SA    \hfill 1. & Lo = Pos  & Lo = Even & SP \\
\hline \bf       \hfill 2. & Lo = Even & SP        & \\
\hline \bf       \hfill 3. & SP        &           & \\
\hline \multicolumn{4}{|l|}{\bf Markierungen (inklusive Trumpffarbe):} \\
       \multicolumn{4}{|l|}{erster freier Abwurf ist Lavinthal} \\
       \multicolumn{4}{|l|}{} \\
\hline
\end{tabularx}

\begin{tabularx}{\columnwidth}{|X|}
\multicolumn{1}{c}{\bf \large Kontras} \\
\hline {\bf Informationskontra (Stil; Antworten; Reopening)} \\
\hline {aggressive Reopenings, auch pre-balancing} \\
       {} \\
       {} \\
       {} \\
\hline {\bf Negativ-Kontra, Kompetitiv-Kontra und weitere} \\
       {\bf (Re-)Kontras} \\
\hline {Negativ-X bis 4♥, darüber Optional} \\
       {zeigt vor allem (andere) OF, UF kann auch kurz sein} \\
       {Bsp: 1♦-(1♠)-X zeigt ♥, nicht unbedingt ♣} \\
       {Support-Kontra und -Rekontra} \\
       {Responsivkontra} \\
       {} \\
       {} \\
       {} \\
\hline \end{tabularx}

\begin{tabularx}{\columnwidth}{|X|}
\hline \multicolumn{1}{c}{\bf \Large Deutsche Konventionskarte} \\
\hline \multicolumn{1}{c}{\bf \Large ♠ ♥ DBV e.V. ♦ ♣} \\
\hline {\bf Kategorie:} B \\
\hline {\bf Club:} BC Bayer Leverkusen - II \\
\hline {\bf Turnier:} Rhein-Ruhr Liga 2012 \\
\hline {\bf Paar:} Michael Goetze -- Christoph Berg \\

\hline \multicolumn{1}{c}{\bf \Large System-Zusammenfassung} \\
\hline {\bf Genereller Stil} \\
\hline 5er Oberfarben \\
 Game tries sind nat \\
 Bergen raises \\
 \\
\hline {\bf 1 SA Eröffnung} \\
\hline 15-17 bal \\
\hline {\bf 2 über 1 Antworten} \\
\hline selbstforcierend \\
\hline {\bf Gebote, die besondere Gegenreizungen erfordern} \\
\hline 2♦ = Wilkosz (Zweifärber 5/5+ mit Oberfarbe) \\
 \\
 \\
 \\
 \\
 \\
 \\
 \\
 \\
 \\
 \\
 \\
 \\
 \\
 \\
 \\
 \\
 \\
\hline {\bf Forcing Pass Sequenzen} \\
\hline \\
 \\
 \\
\hline {\bf Wichtige sonstige Bemerkungen} \\
\hline \\
 \\
 \\
\hline {\bf Bluffs} \\
\hline keine bekannt \\
\hline \end{tabularx}
\begin{flushright}
{\scriptsize \today}
\end{flushright}

\end{multicols}

\begin{tabularx}{\columnwidth}{|c|c|c|c|l|l|X|l|}
\hline
 \begin{sideways}\bf Eröffnung\end{sideways} &
 \begin{sideways}X\,wenn\,künstlich\end{sideways} &
 \begin{sideways}Min.\,Anz.\,Karten\end{sideways} &
 \begin{sideways}Negativ-X\,bis\end{sideways} &
 \bf Beschreibung &
 \bf Antworten &
 \bf Weiterreizung &
 \bf Änderungen als gepasste Hand \\
\hline 1♣  & &3& & 11-21 nat & 2♣ PF Hebung; 2♦ Einl. Hebung; & NMF, 4. Farbe PF & \\
           & & & & & ♥/♠ 6er-Länge, 4-8 Pkt.; 3♣ schwache Hebung & & \\
\hline 1♦  & &3& & 11-21 nat, 4+ Karo oder 4432 & 2♦ PF Hebung; 2♥/♠ 6er-Länge, 4-8 Pkt.; & NMF, 4. Farbe PF & \\
           & & & & & 3♣ Einl. Hebung; 3♦ schwache Hebung & & \\
\hline 1♥  & &5& & 11-21 nat & 2♠ 6er-Länge, 4-8 Pkt., 2SA 4er-Anschluss PF,  & Auf 2SA: 3er Stufe Kürze, 4er Stufe Länge& \\
           & & & & & 3♣ 4er-Anschluss 7-9 Pkt, 3♦ 4er-Anschluss 10-11 Pkt, & & \\
           & & & & & 3♥ 4er-Anschluss 0-6 Pkt. & & \\
\hline 1♠  & &5& & 11-21 nat & 2SA 4er-Anschluss PF, 3♣ 4er-Anschluss 7-9 Pkt, & Auf 2SA: 3er Stufe Kürze, 4er Stufe Länge& \\
           & & & & & 3♦ 4er-Anschluss 10-11 Pkt, 3♠ 4er-Anschluss 0-6 Pkt. & & \\
\hline 1SA & & & & 15-17 ausgeglichen, & 2♣ Stayman (auch ohne 4er OF), 2♦ C\oe{}ur, 2♥ Pik,& & \\
           & & & & auch mit 5er OF & 2♠ Treff, 2SA Karo, 3♣ 4441, 3♦ 4414, & & \\
           & & & & & 3♥ 31(45), 3♠ 13(45) & & \\
\hline 2♣  & &0& & PF oder (semi)ausgeglichen 22-24 & 2♦ relay & 2♣-2♦-2SA wie 2SA EÖ & \\
\hline 2♦  &X&0& & Zweifärber (5-5 oder länger) & 2♥/♠ p/c, 2SA PF relay, 3♦ einl. in OF & Auf 2SA: 3♣ OF+Treff (3♦ relay), 3♦ & In Systemkat. C: W2 Karo\\
           & & & & mit mind. einer OF & 3♥ p/c, 4♣ fragt OF in TRF, 4♦ fragt OF & Karo+C\oe{}ur, 3♥ OFen, 3♠ Pik+Karo & \\
\hline 2♥  & &5& & Weak Two in C\oe{}ur & RONF, 2SA Relay & Auf 2SA min/min/max/max & \\
\hline 2♠  & &5& & Weak Two in Pik & RONF, 2SA Relay & Auf 2SA min/min/max/max & \\
\hline 2SA & & & & 20-21 (semi)ausgeglichen & 3♣ Puppet-Stayman, 3♦ C\oe{}ur, 3♥ Pik  & 2SA-3♣-3♦ 4er OF, 2SA-3♣-3♦-4♦ beide & \\
           & & & & & 3♠ 5er Pik und 4er Cœur & & \\
\hline 3♣  & &6& & Sperrgebot & & & \\
\hline 3♦  & &6& & Sperrgebot & & & \\
\hline 3♥  & &6& & Sperrgebot & & & \\
\hline 3♠  & &6& & Sperrgebot & & & \\
\hline 3SA &X& & & Stehende UF & 4♣ p/c, 4♦ fragt Single & \multicolumn{2}{l|}{\bf Gebote auf hoher Stufe (inkl. Schlemmreizung)} \\
\hline 4♣  & &7& & Sperrgebot & & \multicolumn{2}{l|}{Splinter, RKCB 1430} \\
\hline 4♦  & &7& & Sperrgebot & & \multicolumn{2}{l|}{} \\
\hline 4♥  & &6& & Sperrgebot & & \multicolumn{2}{l|}{} \\
\hline 4♠  & &6& & Sperrgebot & & \multicolumn{2}{l|}{} \\
\hline \end{tabularx}

\end{document}
