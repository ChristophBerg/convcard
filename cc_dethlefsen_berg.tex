%    \begin{macrocode}
\documentclass{article}
\usepackage{miniDBV2,amsthm}
\usepackage[a4paper, landscape, margin=5mm]{geometry}
\usepackage{multicol,tabularx}
\renewcommand{\familydefault}{cmss}
\begin{document}

\newtheorem{convention}{}
\newcommand{\Ref}[1]{$^{\ref{#1}}$}

\newcommand{\Karte}[2]{
\hspace{2.6mm}
\beginMinikarteNeu
	{\large #1}{\large #2}
%
\Grundsystem{\large 5er Oberfarben}
\EinSA{\large15-17}{\large15-17}
%\EinSAkleinesSingle
%\EinSATopSingle
%\EinSAFuenferOFregel
\EinSAFuenferOFselten
\Mindestlaengen{\large3}{\large3}{\large5}{\large5}
\EinTreffBed{3+ \treff}
\EinTreffAnt{Spr�nge schwach}
	{Relaistransfer}
\EinKaroBed{3+ (meist 4+) \karo}
\EinKaroAnt{dito}
	{}
\EinCoeurBed{5+ \coeur}
\EinCoeurAnt{2\SA{} Jacoby (PF, guter Fit), danach 3er-Stufe Werte/2. Farbe, 4er-Stufe Splinter}
	{3\treff/3\karo (im Sprung) Bergen, 3\coeur/4\coeur{} sperrend}
\EinPikBed{5+ \pik}
\EinPikAnt{dito}
	{}
\EinSABed{nat.}
\EinSAAnt{2\treff{} Stayman ab 8F, Transfers ab 0F
	(2\pik{} = \treff, 3\treff{} = \karo)}
	{3\karo/\coeur/\pik Schlemmint., 4\treff{} Assfrage}
%
\ZweiTreffBed
	{bel. Semiforcing, starke SA 22-23}
	{}
\ZweiTreffAnt{2\karo{} Relais, Rest nat}
\ZweiKaroBed
	{bel. Partieforcing, starke SA 24+}
	{}
\ZweiKaroAnt{}
\ZweiCoeurBed
	{Weak Two in \coeur}
	{}
\ZweiCoeurAnt{\SA{} Ogust (min-min-max-max); neue Farbe forcing}
\ZweiPikBed{Weak Two in \pik}
	{}
\ZweiPikAnt{dito}
\ZweiSABed{semiausgeglichen, 20-21F}
	{}
\ZweiSAAnt{3\treff{} Stayman, 3\karo/3\coeur{} Transfer
	}
%
\BesondereZweierUndHoeher{3\SA: Gambling (stehende 7er-UF ohne Nebenwerte)}
	{4\SA: 6-5+ in UF}
%
\InfoKontraAb{12}
\InfoKontraOF
%\InfoKontraWerte
\FarbGegenEiner{8}{16}
\FarbGegenZweier{10}{17}
\StilDerGegenreizung{kompetitiv}
\Weiterreizung{Farbwechsel nonforcing}
\EinSAGegen{16-18F}{11-14F}
	{}
\SprungGegen{Weak Jumps}
	{Michaels Pr�zis}
%
\GegenEinSA
	{Multi-Landy: 2\treff: beide OF, 2\karo: OF-Einf�rber, 2\OF: 5-4+ OF-UF, }
	{\hphantom{Multi-Landy:} 2\SA: beide UF, 3\UF: UF-Einf�rber}
	{\hphantom{Multi-Landy:} X ist Strafe}
\AndereGegenreizungen
	{}
	{}
	{}
%
\SequenzHoechste{}
%\SequenzZweite{}
\InnereSequenzHoechste{}
%\InnereSequenzZweite{}
%\AusspielDritteFuenfte
%\AusspielVierte
\AusspielZweiteVierte
\AusspielSonstiges{Double klein, ab 10x hoch}
\AusspielSA{}
	{}
%\PositivHoch
\PositivNiedrig
%\PositivSonstiges{}
%\GeradeHoch
\GeradeNiedrig
%\GeradeSonstiges{}
\Abwuerfe{direkt}
\MarkierungenSA{Abw�rfe Lavinthal}
	{(kein Smith-Peter)}%{\hspace{87mm} Vereinbarungen im Innenteil $\rightarrow$}
%
\Datum{\footnotesize \today}
%
\endMinikarteNeu

%\begin{twocolumn}

\section*{Vereinbarungen}

\begin{itemize}
\item Help Suit Trialbids
\item Mixed Cuebids
\item Assfrage ist generell RKCB 30/41, meist 4\SA
 \begin{itemize}
  \item dann Zahl der K�nige (ohne Trumpfdame direkt 6)
 %\item DOPI-ROPI (X 30/P 41)
 %\item Exclusion KCB
 \end{itemize}
%\item 1OF-2OF-3OF ist nur ungest�rt einladend (dann unausg., sonst 2SA)
\item Supportkontra, Negativkontras, 3. (Unter-)Farbe forcing
\item Lebensohl
%\item Nach Zwischenreizung
 %\begin{itemize}
 %\item 1x-(1/2y)-2z ist nonforcing
 %\item 1OF-(X)-2SA zeigt Fit, einladend (Truscott)
 %\item 1OF-(X)-3SA zeigt Fit, forcing
 %\item 1OF-(2x)-3x zeigt Fit
 %\item 1UF-(2x)-3x fragt nach Stopper
 %\end{itemize}
\end{itemize}

%\end{twocolumn}
}

\newcommand{\Doppelkarte}[2]{
\Karte{#1}{#2}
\Karte{#2}{#1}
}

\Doppelkarte{Frank Dethlefsen}{Christoph Berg}

\end{document}
%    \end{macrocode}
