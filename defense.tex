%
% Gegenreizung und Verteidigung
% 5er-OF (cc_walsh.tex) und Moscito (cc_moscito05.tex)
%
\InfoKontraAb{12 FP}
\InfoKontraOF
%\InfoKontraWerte
\FarbGegenEiner{8}{15}
\FarbGegenZweier{10}{16}
\StilDerGegenreizung{in der Regel konstruktiv, schwache Hebungen}
\Weiterreizung{Farbw.\ nonf., �berr.\ fr.\ Stopper ($\rightarrow$\SA),
	zeigt Fit/stark Einf.}
\EinSAGegen
	%{15-18 FP}
	{Raptor}
	{11-14 FP}
	{relativ ausgeglichen, Stopper, Weiterreizung wie nach 1 \SA-Er�ffnung}
\SprungGegen{Weak Jumps: schwach, sperrend}
	{Michaels Pr�zis:
	(1UF)\ra 2\karo: \OF, 2\SA: aUF/\coeur;
	(1OF)\ra 2OF: \treff/aOF, 2\SA: \UF, 3\treff: \karo/aOF}
%
\GegenEinSA{Multi-Landy: X: 5-4 UF-OF, 2\treff: 5-4\pl \OF\ (\ra 2\karo: gleiche L�nge in \OF),
	2\karo: OF-Einf�rber,}
	{\hphantom{Multi-Landy:} 2OF: 5-4\pl OF/UF, 2\SA: 5-5\pl in \UF, 3UF: UF-Einf�rber}
	{gegen schwachen \SA: X: mind. gleiche St�rke (oberes Ende Punktspanne), Rest wie oben}
\AndereGegenreizungen
	%{Starke 1\treff \ra Crash}
	{Starke 1\treff \ra Multi-Landy (1\SA: UF, X: 13-15 ausgegl.)}
	{2\karo-Multi \ra X: Info-X gegen \coeur-Weak Two}
	{Lebensohl und Scrambling 2\SA\ nach Info-X;
	1\SA/2\SA\ als Zweif�rber in 4. Hand; Kompetitive 2\SA}
%
\SequenzHoechste{}
%\SequenzZweite{}
%\InnereSequenzHoechste{}
\InnereSequenzZweite{}
\AusspielDritteFuenfte
%\AusspielVierte
%\AusspielZweiteVierte
\AusspielSonstiges{Double hoch}
\AusspielSA{4.-h�chste, 10/9 verspricht 0 oder 2 h�here}
	{}
%\PositivHoch
\PositivNiedrig
%\PositivSonstiges{}
%\GeradeHoch
\GeradeNiedrig
%\GeradeSonstiges{}
\Abwuerfe{Direkte Marke, Figur zeigt Karte darunter}
\MarkierungenSA{Lavinthal, Smith Peter}
	{}
%
