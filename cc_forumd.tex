%    \begin{macrocode}
\documentclass{article}
\usepackage{miniDBV,amsthm}
\usepackage[a4paper, landscape, margin=5mm]{geometry}
\usepackage{multicol,tabularx}
\renewcommand{\familydefault}{cmss}
\begin{document}

\newtheorem{convention}{}
\newcommand{\Ref}[1]{$^{\ref{#1}}$}

\hspace{2.6mm}
\beginMinikarteNeu
	%{\large Christoph Berg}{\large Frank Luithle}
	{\large Alexandra Wipper}{\large Christoph Berg}
	%{}{}
%
\Grundsystem{\large 5er Oberfarben (Forum D)}
\EinSA{\large15-17F}{\large15-17F}
%\EinSAkleinesSingle
%\EinSATopSingle
%\EinSAFuenferOFregel
%\EinSAFuenferOFselten
\Mindestlaengen{\large3}{\large3}{\large5}{\large5}
\EinTreffBed{min. 3er, 12+F}
\EinTreffAnt{2/3/4/5\treff: Limit-Hebung, 2\OF: 6er 5-8F,
	}
\EinKaroBed{min. 3er (4er au"ser bei 4432-Verteilung), 12+F}
\EinKaroAnt{2/3/4/5\karo: Limit-Hebung, 2\OF: 6er 5-8F}
\EinCoeurBed{min. 5er, 12+F}
\EinCoeurAnt{2/3/4\coeur: Limit-Hebung, 2\pik: 6er 5-8F
	}
\EinPikBed{min. 5er, 12+F}
\EinPikAnt{2/3/4\pik: Limit-Hebung
	}
\EinSABed{ausgeglichen, 15-17F}
\EinSAAnt{2\treff: Stayman (8+F; 2\SA: beide OF), 2\karo/\coeur: Transfer (0+F),
	4\treff: Gerber (04/1/2/3)}
%
\ZweiTreffBed
	{Partieforcing, \SA\ 22+F}
	{}
\ZweiTreffAnt{2\karo: Relais}
\ZweiKaroBed
	{Semiforcing in \karo}
	{}
\ZweiKaroAnt{nat"urlich}
\ZweiCoeurBed
	{Semiforcing in \coeur}
	{}
\ZweiCoeurAnt{nat"urlich}
\ZweiPikBed{Semiforcing in \pik}
	{}
\ZweiPikAnt{nat"urlich}
\ZweiSABed{ausgeglichen, 20-21F}
	{}
\ZweiSAAnt{3\treff: Stayman (4+F), 3\karo/\coeur: Transfer f�r \OF (0+F),
	}
%
\BesondereZweierUndHoeher{}
	{}
%
\InfoKontraAb{12F}
\InfoKontraOF
%\InfoKontraWerte
\FarbGegenEiner{8}{15}
\FarbGegenZweier{10}{16}
\StilDerGegenreizung{in der Regel konstruktiv}
\Weiterreizung{Farbwechsel nonforcing, �berruf fragt nach Stopper}
\EinSAGegen{15-18 F}{11-14 F}{relativ ausgeglichen, Stopper, Weiterreizung wie nach 1 \SA-Er�ffnung}
\SprungGegen{Weak Jumps: schwach, sperrend}
	{}
	%{Michaels Pr�zis:
	%\UF\ $\rightarrow$ 2\karo: \OF, 2\SA: aUF+\coeur;
	%\OF\ $\rightarrow$ �berruf: \treff+aOF, 2\SA: \UF, 3\treff: \karo+aOF}
%
\GegenEinSA
	{nat"urlich}{}{}
	%{Multi-Landy: X: 5-4+ in UF-OF, 2\treff: 5-4+ \OF\ (2\karo: gleiche L�nge in \OF),
	%2\karo: OF-Einf�rber,}
	%{\hphantom{Multi-Landy:} 2\OF: 5-4+ \OF-UF, 2\SA: 5-5+ in \UF, 3\UF: UF-Einf�rber}
	%{gegen schwachen \SA: X: mind. gleiche St�rke}
\AndereGegenreizungen
	{}{}{}
	%{gegen 2\karo-Multi: X: Info-Kontra gegen \coeur-Weak Two}
	%{Lebensohl nach Info-X; 2\SA\ als Zweif�rber in 4. Hand}
	%{}
%
\SequenzHoechste{}
%\SequenzZweite{}
%\InnereSequenzHoechste{}
\InnereSequenzZweite{}
\AusspielDritteFuenfte
%\AusspielVierte
%\AusspielZweiteVierte
\AusspielSonstiges{Double hoch}%, K $\Rightarrow$ L�ngenm.}
\AusspielSA{4.-h�chste}
	{}
%\PositivHoch
\PositivNiedrig
%\PositivSonstiges{}
%\GeradeHoch
\GeradeNiedrig
%\GeradeSonstiges{}
\Abwuerfe{Direkte Marke}
\MarkierungenSA{Lavinthal}
	{}
%
\Datum{\today}
%
\endMinikarteNeu

\newpage

\setlength{\parindent}{0pt}
\setlength{\parskip}{1ex}
%\setlength{\columnseprule}{.4pt}

\begin{multicols*}{3}

\section*{Konventionen zum System}

\begin{convention}[Schlemmkonventionen] \label{schlemm}
$\circ$ Mixed Cuebids \\
$\circ$ 4\SA: Blackwood 0/4, 1, 2, 3 Asse \\
$\circ$ 5\SA: 0/4, 1, 2, 3 K"onige \\
\end{convention}

\end{multicols*}

\end{document}
%    \end{macrocode}
